% Template:     Informe LaTeX
% Documento:    Archivo principal
% Versión:      8.2.4 (29/04/2023)
% Codificación: UTF-8
%
% Autor: Pablo Pizarro R.
%        pablo@ppizarror.com
%
% Manual template: [https://latex.ppizarror.com/informe]
% Licencia MIT:    [https://opensource.org/licenses/MIT]

% CREACIÓN DEL DOCUMENTO
\documentclass[
	spanish, % Idioma: spanish, english, etc.
	oneside
]{article}

% INFORMACIÓN DEL DOCUMENTO
\def\documenttitle {Proyecto 2}
\def\documentsubtitle {Multihebra Sincronización}
\def\documentsubject {Programación multihebra y mecanismos de sincronización}

\def\documentauthor {Artigues & Meza & Soto}
\def\coursename {Sistemas Operativos}
\def\coursecode {503308}

\def\universityname {Universidad de Concepción}
\def\universityfaculty {Facultad de Ingeniería}
\def\universitydepartment {Departamento de Informática y Ciencias de la Computación}
\def\universitydepartmentimage {departamentos/fiudec2}
\def\universitydepartmentimagecfg {height=1.57cm}
\def\universitylocation {Concepción, Chile}

% INTEGRANTES, PROFESORES Y FECHAS
\def\authortable {
	\begin{tabular}{ll}
		Integrantes:
		& \begin{tabular}[t]{l}
			Roberto Felipe Artigues Escobar \\
			Emilio Juan Meza Quiroz \\
			Nicolas Eduardo Soto Soto
		\end{tabular} \\ & \\
		Profesora:
		& \begin{tabular}[t]{l}
			Cecilia Paola Hernandez Rivas
		\end{tabular} \\
		% Auxiliar:
		% & \begin{tabular}[t]{l}
		% 	Auxiliar 1
		% \end{tabular} \\
		% Ayudantes:
		% & \begin{tabular}[t]{l}
		% 	Ayudante 1 \\
		% 	Ayudante 2
		% \end{tabular} \\
		% \multicolumn{2}{l}{Ayudante de laboratorio: Ayudante 1} \\
		& \\
		% \multicolumn{2}{l}{Fecha de realización: \today} \\
		\multicolumn{2}{l}{Fecha de entrega: 24 de Noviembre de 2023} \\
		\multicolumn{2}{l}{\universitylocation}
	\end{tabular}
}

\DeclareUnicodeCharacter{0301}{*************************************}

% IMPORTACIÓN DEL TEMPLATE
\input{template}

% INICIO DE PÁGINAS
\begin{document}

% PORTADA
\templatePortrait

% CONFIGURACION DE PÁGINA Y ENCABEZADOS
\templatePagecfg

% RESUMEN O ABSTRACT
% \begin{abstractd}
% 	% \lipsum[1] % Párrafo ejemplo, se puede borrar
% 	Se realizó un trabajo experimental en el cual se aplicaron cargas de distintos pesos a una armadura de tipo Warren. Durante el proceso, se midieron los valores de deformación utilizando un medidor de deformaciones y se mantuvieron registrados mientras se mantenían las cargas aplicadas. \\ \\
% 	A continuación, se multiplicaron los valores obtenidos por la constante 𝛼 con el fin de obtener los valores de fuerza correspondientes. Además, se pudo determinar si las barras de la armadura estaban sometidas a tracción o compresión.
% 	\\ \\
% 	Los objetivos del trabajo fueron:  
% 	\begin{itemize}
% 		\item Estudiar el equilibrio de fuerzas en una estructura.
% 		\item Medir fuerza axial en miembros estructurales.
% 		\item Hacer una breve descripción del banco de ensayos, condiciones de carga.
% 		\item Indicar el diagrama de cuerpo libre de la armadura.
% 		\item Realizar una comparación y discusión de los resultados.
% 	\end{itemize}

% 	\noindent Con respecto a los resultados obtenidos no fueron los esperados, tanto en magnitud de fuerzas como en estados de carga. 
% 	Esto pudo haber ocurrido por fallos en los materiales o por errores al momento de medir las cargas. \\

% 	\noindent En conclusión, es posible afirmar que el análisis teórico solo representa una aproximación a la realidad y no tiene en cuenta todos 
% 	los factores que podrían influir en los resultados de la experimentación en condiciones reales, por lo tanto, cabe la posibilidad 
% 	de que los resultados del análisis teórico y experimental no sean iguales.
	 

% \end{abstractd}

% TABLA DE CONTENIDOS - ÍNDICE
% \templateIndex

% CONFIGURACIONES FINALES
\templateFinalcfg

% ======================= INICIO DEL DOCUMENTO =======================

\section*{Detalle de la Implementación}

\noindent Este proyecto se ha implementado con el objetivo de analizar y comparar dos métodos distintos de sincronización en programación multihebra:
 el uso de semáforos y el uso de mutex combinado con variables de condición. 
\vspace*{12pt}

\noindent El programa está escrito en C++ y hace uso extensivo de las bibliotecas estándar, incluyendo \texttt{<thread>}, \texttt{<mutex>}, \texttt{<semaphore>}, y \texttt{<filesystem>}, para gestionar hilos, sincronización y operaciones de archivos.

\vspace*{8pt}
\noindent El flujo de trabajo del programa es el siguiente:
\begin{enumerate}
    \item \textbf{Lectura de Parámetros}: El programa comienza leyendo los parámetros de entrada que incluyen el directorio de los archivos de genomas, el valor umbral del contenido genético CG y el modo de sincronización (semaforos o mutex).
    \item \textbf{Procesamiento Multihebra}: Para cada archivo en el directorio especificado, se crea un hilo que procesa el archivo de manera independiente. Esto implica calcular el contenido genético CG de todo el archivo.
    \item \textbf{Sincronización y Almacenamiento en Cola}: Dependiendo del modo seleccionado, el programa utiliza semáforos o mutex con variables de condición para controlar el acceso a la cola compartida. Si el contenido genético CG de un archivo supera el umbral, su nombre se almacena en la cola.
    \begin{itemize}
        \item En el modo de semáforos, se usa \texttt{std::counting\_semaphore} para asegurar que solo un hilo a la vez pueda acceder a la cola.
        \item En el modo de mutex, se emplea \texttt{std::mutex} junto con \texttt{std::condition\_variable} para una gestión segura de hilos que acceden o modifican la cola.
    \end{itemize}
    \item \textbf{Salida}: Una vez que todos los hilos han terminado su ejecución, el programa procede a vaciar la cola compartida, imprimiendo los nombres de los archivos que cumplen con el criterio del umbral del contenido genético CG.
\end{enumerate}

\newpage
\section*{Evaluación Experimental}
\noindent Para obtener los resultados se utilizó Linux en WSL2. Se realizaron 15 ejecuciones para ambos modos de sincronización con 10 archivos de genomas y se obtuvo su promedio. 
\\ Los comandos utilizados para ejecutar las pruebas fueron:

\begin{verbatim}
/usr/bin/time -v ./main.exe Genomas 0.6 1
/usr/bin/time -v ./main.exe Genomas 0.6 2
\end{verbatim}

\vspace*{5pt}
\noindent Los resultados se muestran en la siguiente tabla:
\vspace*{2pt}

\begin{table}[h]
\centering
\begin{tabular}{|l|l|l|}
\hline
\textbf{Modo} & \textbf{Tiempo (s)} & \textbf{RAM (KB)} \\ \hline
Semáforo & 1.136 & 4291.2 \\ \hline
Mutex & 1.152 & 4342.4 \\ \hline
\end{tabular}
\caption{Comparación de tiempo y uso de memoria}
\label{tab:comparacion}
\end{table}

\section*{Interpretación de Resultados}

\noindent Los resultados obtenidos de la evaluación experimental revelan diferencias mínimas en el rendimiento entre los dos métodos de sincronización. En términos de tiempo de ejecución, el modo semáforo registró un promedio de 1.136 segundos, mientras que el modo mutex mostró un promedio ligeramente superior de 1.152 segundos. Esta pequeña diferencia sugiere que ambos métodos son casi equivalentes en eficiencia de tiempo para la tarea dada.
\vspace*{10pt}

\noindent En cuanto al uso de memoria, el modo semáforo utilizó en promedio 4291.2 KB, frente a los 4342.4 KB del modo mutex. Aunque la diferencia es igualmente pequeña, indica una ligera ventaja en eficiencia de memoria para el modo semáforo.
\vspace*{10pt}

\noindent En conclusión, mientras que existen diferencias leves en tiempo y uso de memoria entre los dos modos, estos son marginalmente significativos, indicando que ambos enfoques son prácticamente comparables en términos de rendimiento para el procesamiento de archivos de genomas en el contexto del experimento.





% FIN DEL DOCUMENTO
\end{document}